\documentclass[uplatex]{utbook}
% Settings
\usepackage{pxrubrica}
\obeylines
\parindent = 0pt
%
\begin{document}
朝
――ミーンミーン。蝉の主張と木のざわめきが全ての音をかき消している。夏の太陽が殺人的な日差しを照り続け、地面を焦がしている。
「行ってくる」  
蝉の音にかき消されるような小さな挨拶をしつつ、\ruby{緒岸}{おき|し} \ruby{蒼}{アオ}は玄関の扉を開けようとした。 
「今日夜、話があるから」  
祖父が後ろから話しかけてきた、咄嗟に蒼は後ろを向いた。
「わかった」  
それ以降の会話はなく蒼は再び「行ってくる」と叔父に挨拶を残し、外に出た。
外は恨めしいほどの快晴で、朝の程よい風が吹き抜けている。風はあるがアスファルトが焼けるようにあつい。
緒岸の家は他の生徒の通学路から少しそれた場所にある。なので登校時の最初は常に一人だ。彼自身は一人が好きなのでこの道は気に入っている。
もう少し日陰があればなとは思うのだが。
数分ほど昨日の出来事などとりとめのない記憶を思い出しつつ一人で登校していると賑やかな通学路が見えて来る。 
数名の男子、女子のグループが8、9ほど一つの学校を目指し歩いている。この時間帯は電車通学の人たちがこの道を通るので比較的賑やかだ。
「昨日の動画見た?」「見た見たやばいあれ」「また人が消えたんだって」「今度はだれ?」「週末どこ行くよ」
蝉の鳴き声をかき消すような雑踏の中、一人で歩いている。会話がたくさん耳に入って来るが緒岸にとってそれは心地の良いものだった。
通学路はいくつかの交差点を超え、最後に長い直線がある。
直線ゆえまだ学校まで距離があるが校舎がうっすらと見える。この直線にはいればもう少しで学校だ。そして直線に入るといつも必ず、
「よう、課題終えた?昨日の問題結構量多かったじゃん」
「一応」
後ろから気さくな声で話しかけられる
緒岸は立ち止まり、声の主が隣に来るのを待つ。声の主は軽快な足取りで前に出てきた。
メガネをかけたその学生は手を合わせ申し訳なさそうに緒岸に言った。
「最後の問題意味不明だったから教えて」
「黒田お前、前回もそんな感じでめんどくさい問題だけ意味不明と言ってたよな?」
「気のせいでしょ、気のせい」
黒田と呼ばれた学生は緒岸の隣に並ぶ。基本的に同じ場所、同じ時間で合流する。最初は離れるだろうと思っていたが1年以上も続くのは驚きだ。
二人で登校する事で課題の提出率が上がる、今日の内容の確認ができる、を目的に始めたが最近は専らどちらかがどちらかの課題を
写すような要求か、学校と関係ない話で盛り上がることが多い。
合流してからしばらく歩き続けるとくたびれた校舎が見えてきた。最近一部を改築するなんて話があるが、真偽のほどが定かではない。
一応この周りでは歴史がある方らしいと耳に胼胝(タコ)ができるほど緒岸は親戚から聞かされていたが、崩壊する前に新しくしてくれと思うばかりである。
校舎に入り、廊下を歩く。人が行き交う廊下の中、今日も一日がんばりますか、と教室に二人は入り準備を始めた。

----
昼の暑さが落ち着き、心地よい風が教室を吹き抜ける中、最後の鐘がなった。 
HRが特に何かあるわけではなく、明日の予定などをを淡々と先生が話している。強いて言うなら最近あった失踪事件について注意があった。
先ほどの淡々とした話し方から打って変わって抑揚をつけながら話し始めた。聞いてほしいと言う思いを緒岸は感じた。
「失踪事件てどう思う?あり得ると思う?」
先生の努力にも目をくれず、HR中黒田が声を絞って囁いてきた。
「まぁ……、何かはあるよね、先生たちが怪しいけど」
「だよなぁ、担任の様子を見るに一部の先生のみ加担してそうだけど」
「確かに、人数的にも結構人で入りそうだけどね」
怒られないギリギリの声量で二人は軽く話し合う。ギリギリの声量とはいえ先生にはバレてしまい、中断させられてしまった。
では、
「終わります」
先生の号令で今日の集団活動は幕を閉じた。
生徒が一斉に羽を伸ばす。各々の部活へ行く生徒、帰宅する生徒皆散り散りになる。
「今日用事があるから帰るわ」
「うい」
黒田は用事があるらしくそそくさと教室の外へ歩き出した。緒岸はひらひらと手を振り申し訳程度に見送った。
用事が何かは気になるといえば気になるが聞いたところで特に何かあるわけではないよなと基本的に緒岸は干渉しない。
別れた後、緒岸はいつものようにある部屋へと歩き出した。

目的地は校舎の端っこの方にある、コンピュータ演習室である。
緒岸はIT研究部と言う部活に所属している。研究部と言うのは名前だけのもので、主な活動は物を作って適当に公開していくと言うとても緩いものだ。
昔は大会に参加する、何かに展示する作品を作る等積極的に外に向けて発表していたらしいが今はそのような活動はやっておらず、皆好きなものを作って
適当に公開するだけになった。緒岸は前の部長になぜこうなったかを聞いたのだが、主な理由は大会の日に遅刻する人が多いので出なくなった、と言うことらしい。
顧問の先生も特に協調性を重視するタイプでもなく、大会にでることないまま次の世代に写ってしまい文化が途切れたと言うわけだ。
入部時は今の7倍ほどいたが日に日に来なくなり、ほとんどが幽霊部員である。
たまにあるイベントや、勉強会などにだけ来る人が半分程度で、緒岸も全員の顔を認識していない。
緒岸はそのIT研究部の部室であるコンピュータ演習室に向かって歩き続けている。
演習室に近づくにつれて特別教室が増え、廊下の雰囲気が段々と淋しくなっていく。ここら辺の廊下になって来ると電気も消えているので日光のみを頼りに廊下を歩く。
遠くの方で楽器の音や掛け声などの音が聞こえるが、自分の周りには人一人もいない。あるのは薄暗い中赤く揺らめく消火栓ぐらいである。
演習室の前に着き、古ぼけた心もとない鍵付き扉が出迎える。鍵は演習室脇の消火器箱の中に隠されている。これは部員が顧問のところに行かなくてもいいと言う顧問の配慮だ。
多分顧問も面倒だと思っている故、利害の一致があるからだと緒岸はほんのり思うが、便利なのでそっとしておく。
鍵を使い扉を開け中に入る。
窓はあるが他の校舎の影になっており少し薄暗い。掃除は毎日行われているらしいがそこはかとなく埃臭い。
薄汚れた空間に古いモデルのパソコンが40台ほど設置されている。そこまで新しいわけではなく最近はコンピュータが必要と騒がれているが自分の学校は大丈夫かななど余計な心配をしてしまう。
いつものように奥の方のスペースに座る。特にこの席に愛着があるわけではないがこの席が一番扉から離れているので厄介ごとに巻き込まれる可能性が
低いと考え、この席を毎回利用している。席について一息ついたところで鞄からコンピュータを取り出した。高校の規則でコンピュータについて規制されてないので
理由に持ち込める。この高校の好きなポイントの一つだ。
「昨日どこまで進んだっけ」
そう呟きながらをコンピュータを開いた。今朝の作業画面が表示される。今作ってるのはWebページ用のライブラリだ。
実装途中の部分を完成しないとなぁと思いキーを叩く。
カタカタ……、キーを叩く音と、たまに発生する沈黙。
静寂の中、自分のキーの音のみが場を支配する状態。この状態の時一番脳を使えると思っているので緒岸はこの時間が好きだった。
薄暗く埃臭いが誰にも邪魔されない、小さな城はものを作るにはもってこいの空間だ。

比較的簡単な開発がひと段落付き、少し難しい問題に直面した。苦い顔をしていると扉が開く音が教室に響く。
「おっつー、遅くなったー」
数少ない”活動的な”IT研部員が静寂を破る。緒岸は作業を中断し扉の方からこっちに歩いて来る短髪少女を見た。
美澄と言う名前の”活動的な”IT研部員は緒岸の隣に来るとコンピュータを鞄から取り出し、鞄を机の上に叩きつけた。
中には特に何も入っているわけでもなく軽く跳ねて机の上で静止した。
「お前、行儀悪いな」
「壊れない力で投げてるから問題ないわ、進捗どう?」
「ぼちぼち」
ここで変に突っ込んで長話になるのは面倒だな、と二人の認識が一致している結果、基本的に挨拶はとてもさっぱりしている。
カタカタカタカタ……、2倍となったキーを叩く音。沈黙の時間が大きく減ったがこれといって雑談があるわけでもない。
「あの部分のレビューお願い :bow: 」
「あのバグは昨日参考になりそうな記事見つけたよ」
二人の間でたまに発生する会話はこのような情報交換の側面しかない。
薄暗い部屋の一室で淡々と開発を続けていくチーム。開発中は互いが互いに対してあまり鑑賞されたくないのでコミュニケーションがあまりない。
「今日は黒田来ないんだっけ」
「あぁ、今日は用事があるらしい」
一応黒田も”活動的な”IT研部員である。今の開発物も緒岸、美澄、黒田の3人で作っている。

----
「今日は終わりにしよう」
窓の外が橙色から黒になったところで、緒岸がそう呟いた。
気がつくと教室の外、遠くから聞こえる他の部活の音が聞こえなくなっていた。
巡回の先生がそろそろ来る頃だろう。怒られるのが面倒だ、そう言いながら二人は早々と帰る準備を始めた。
コンピュータをカバンにしまうと電気を消し、教室を出た。教室は静寂に包まれた。鍵をいつもの消火器箱の中に突っ込んで今日の活動は終わりとなる。
廊下は右も左もわからないぐらい真っ暗だ。多分コンピュータ室に人間がいることを知らず誰かが消灯してしまったのだろう。
緒岸はいつもの事だと、手慣れた様子で携帯のライトのスイッチを入れた。暗い世界の中一つの明かりが道を照らす。二人は明かりを頼りに廊下を歩きはじめた。
「そういえば、噂って聞いたことある?」
数秒暗闇の中沈黙を保っていたが、気まずいのかそっと美澄が呟いた。
沈黙があまり好きでないのだろう、緒岸は会話を終わらせないように当たり障りのない返事をした。
「あれでしょ?人が消えるやつ。まだ帰ってきてないんだってね」
「そうそれ! 」
「実際あれって先生が軟禁かなんかしてるんじゃないかって思うけど」
「そんな事はどうでもよくてっ……」
美澄は何かを言いたそうに口に手を当てて、立ち止まった。
緒岸もつられて立ち止まる。ライトの向きを前から美澄の方に向けた。
「面白いのが、で消えたと思われている場所が、この校舎のこの隣の棟、階段だって話」
美澄は息を語気を荒げながら話す。どこからそんな情報を持って着たんだ、半ば呆れたような顔で緒岸が頷く。
間をおかず話を続ける。声のトーンも少し上がってきた。
「人が消えてきて一人になったところで消えるらしいよ」
「……でな、な、なんと、時間もだいたい今頃! 」
目を輝かしながら話す。緒岸には次の台詞が頭に浮かんだ。彼女の性格を考えるにどう転んでもめんどくさい事になる未来しかない。
「なるほど」「いか「いかないかな」」
緒岸は相槌を打ちつつも行かせないよう食い気味に意思を表明した。会話を打ち切られ美澄はむくれている。
美澄を置いてきぼりにしよう、逃げるのが得策だと思い緒岸は前に歩き始めた。ライトを前の方に向けたので美澄が闇の中に消える。
少し歩くと明るい廊下が見えてきた、下校しようとしている生徒もちらほらといる。普通の学校に戻ってきた気分だ。
後ろから美澄が走ってくる音がした。さすがに明かりがない状態だったので小走りだったが、明るい廊下に入る前に合流した。
「えー? なんでいかないの? こういうの好きじゃん」
全身を使い訴えかける。美澄はなんとしてでも行きたいらしく”行かないなんてなんと常識知らずなんだ”、理不尽な事を言ってくる。
「まず、さっきも言ったように何人も消えてるっていう事実はあるし、何かしらあると思う。監禁とか、拉致とか、あまり安全に帰れる未来が見えない」
緒岸はめんどくさい事があまり好きではない、冷静に諭した。
「いや、幽霊でしょ、普通に考えて」
つまんなそうな物を見る目でこちらを見て来る。手をぶらぶらさせながらこれだからコンピュータはとか言ってるのが緒岸の耳に入ったがそれでも動かない。
しかし彼女はこのママ引き下がらない事を緒岸は知っている。落とし所を提案せねば、さてどうしたものか。そう思いながら緒岸は下駄箱から靴を取り出す。
彼女も何かを考えながら渋々下足に履き替える。呆れ顔でそれを見ていると彼女が突然話し始めた。
「なら準備、準備してなら大丈夫でしょ!? 拉致、監禁等を考えた装備で行けば良いじゃない! 」
これでどうだと胸を張り、決定でしょ? という視線をぶつけて来る。緒岸はキョトンとした顔で美澄をみる。今がチャンスと畳み掛ける。
「なんのためのITよ? こう言う時使うのがITでしょ? 」
これには流石に反論を入れる。
「なるほど? でも装備とはいえ、金かかるんじゃ?あまりお金は使いたくないなぁ」
「大丈夫よ!装備って言ってももう持ってる物を使うわ、例えば前買ってた変なガジェット! あの自分の生存、位置情報を電池が切れるまで
送り続け切れたらすごい事になるとか言ってた変な最強防犯キーホルダーとか」
緒岸の心の中では葛藤が起こっていた。めんどくさい事になると思いつつ、この前買ったガジェットの事を考える。
確かにあのガジェット(変ではない)一日使っただけでまだ非常事態の実験をしていない。
そもそも校舎のセキュリティや街の治安を考えてもどうせ大事になるはずない。ガジェットのテストはありだな。
歩きながら思考を巡らす。だいたい提案を飲んでいるがここで提案を飲むのは負けた気がする。
そう思い緒岸は後半思考を巡らす”ふり”をしながら通学路を歩く。だんだん生徒が別れて行き閑散とした通学路になってきた。
そもそも時間が時間であり、気がつくと暗い夜道彼女と二人しかいなかった。
彼女と別れる地点に着くと、彼女が目をキラキラとさせながら振り返ってくる。自分のできる限りの演技で、頑張って了承した空気感を出す。
「わかった。近いうちに作戦を練ろう」
「さっすが〜。じゃあ準備しましょうか」
飛び跳ねるほど嬉しいのが全身から伝わってくる。珍しく手を振りながら別れの挨拶をしてきた。そっと緒岸も手を振り返し、別れた。
緒岸は自分の心に一抹の期待感があるのを感じた。

----
「ただいま」
緒岸が玄関を開けた時、珍しく祖父と目があった。
「おう、帰ったか、夕食を食べたら俺の部屋に来い」
「はい」
緒岸は忘れていたのではっとした。たまに話があるからと言ってくる事はあるが、2度も念押しをされるなんて事は結構稀だ。
遺産? 勉強? 思考を巡らし様々な可能性を考える。悪い話か、良い話かを考えていると部屋の外から
「ご飯できたよ」
と思考を中断する声が聞こえた。考えるだけ無駄か、と蒼は夕食を取りに部屋を出た。
食卓に着くと母のみで祖父の姿はなかった。外にいるらしい、先ほど玄関であったのは偶然外にでるタイミングだったと言うだけだった。
特にへんな話ではないのではないかと少し安心し、箸が進む。今日の魚もいつもと同じで美味だ。
夕食を食べ終え、片付けを終えるとそのまま祖父の部屋へ向かった。 
いつもは高く閉じられている襖を開ける。緒岸がこの部屋に入るのは何年ぶりだろうか。
幼少の頃は何度か入った記憶があるが、遠い過去の話であり今の状態は全くといっていいほどわからない。
特に入るなと言われているわけではないが、入る理由がないというだけで気がついたらこの歳になってしまったというだけである。
中に入ると畳と古臭い匂いが鼻をかすめた。中央の明かりが暗い空間を照らしているが、部屋全体を明るくするには
力不足を感じざるおえない。壁の真下などは明かりが届かず、真っ暗だ。緒岸は真っ暗な中に何かある事を感じ取った。
目を凝らして見るとどこで買ったか推測できない謎の置物がすらっと壁に沿って置いてありほんのりと自己主張していた。
祖父は奥の方にある押入れを開け、何か探している。手を動かすたびに埃が宙を舞う。何年間も開けてないことが推測できる。
蒼は置物一つ一つを軽く突いたり手にとって見たりしながら祖父を待った。
「よくこんなガラクタ集めたね。これなんかいる? 」
縄文土器のようだが色使いが現代チック、全体的に曲線が強調されている置物をつまみながら苦笑してる。
「それは昔、どこだっけな? 忘れたが何かの厄除けの効果があるらしい」
ふぅんと緒岸は興味なさそうにそれを置いた。「あった」祖父は埃まみれの小さい木箱を押入れから引っ張り出してきた。
大きさはサッカーボールほどだろうか、正方形で上蓋が付いているだけだった。
「これだこれ」
取り出した木箱を埃を払い明かりの下に置いた。祖父は少々疲れた顔をしているがにやりと口元を歪めて箱の後ろに座る。
蒼も木箱の前に座った。二人は木箱を中心に対になって座っている。
「で、どうしたの? 」
「お前って、幽霊とか信じるタイプだっけ?」
祖父が突拍子もない事を質問してきた。蒼は驚きを隠しつつ、冷静に今の現状を理解しようとした。
宗教。
多分怪しい宗教勧誘でも受けて、何か買わされたのかなと推測した。
ただこのままペースに乗せられると一家壊滅になるかもしれない、それはまずい。蒼は使命感を感じつつ反論をする。
「いや、そういう事はあまり信じないようにしているんだけど」
「そうか」
叔父が箱を開ける。中に拳ぐらいの石があった。綺麗に磨かれており、そこらへんで拾ったものではない事は容易に想像できる。
「実はな、話さねばならいないことがあってな」
祖父は石を手に取り続けた。広げられた手の平に石は置かれている。掴んでいるわけではないので突かれたら落ちるだろう。
「この石を見ていてくれ、今からこれを動かす」
そう言うと耳に届かない程度の絞った声で何かを呟き始めた。
口の動きが止まった瞬間、石が5センチほど手の平から浮かび上がった。
無風であり石に何か付いてる様子ではない。純粋に石が浮かび上がっテイル。
「え、何これえ、何これ、は」「どんなトリック? マジック? 」「いやいやこれは」
緒岸は驚きのあまり立ち上がった。動揺し、質問を何度も叩きつけた。
祖父は動じる事なく石を浮かせ続けている。そして淡々と口を開いた。
「驚くのも無理はないか」
手を引っ込めた、石はそのままその空間に固定されてるようだ、ビクともしない。まるでそこに透明な台でもあるかのごとく
自然に置かれていた。
「いやこんなのありえないじゃん、仕組みは?仕組み」
蒼は自分の理解を超えたと認識し汗がひたいから流れ落ちるのを感じ取った。
動揺は止まない。
「ないんだ、強いて言えば、”神の力”」
「宗教に傾倒して、物を実際に浮かした話なんて聞いたことのないぞ」
納得できない現実に青は畳み掛けて質問を続ける。声量と勢いが上がる。祖父が興奮する蒼を制止した。
「落ち着け、今はそういうものとして欲しい。世界とはそういうものだ、まだ続きがあるそのあとに質問は受け付ける」
「でも、磁力もなにもない状態でものが浮くなんて……」
「多分だがお前の経験と、知識では絶対に腑に落ちないと思う、”十分に発達した科学技術は、魔法と見分けがつかない”という言葉もあるしな」
「なるほど。とりあえずなんか新しい技術かなんかなんだな? 」
半信半疑だったが祖父はまだ話したいことがありそうだったのでとりあえず頷く。
「さて、お前は今石が浮くのを見ただろ?なので今のお前なら見えるはずだ」
祖父は紫の風呂敷を広げた。風呂敷はところどころ黒ずんでいたり綻んでいたりとかなり古い。風呂敷が静止し、何かを唱えている。
唱えた瞬間光が部屋を包み込み目を開けたら、彼女が立っていた。
『状況把握、最適化をしています。お待ちください』
黒く長い髪をもつ彼女は緒岸と同じぐらいの年齢に見える、生気を感じさせないまま、淡々と話し始めた。
なぜか緒岸の通っている制服を着ている。
「最近のは随分と優しくなったな」
祖父は嬉しそうな口調で彼女を見上げた。
『初めまして、みなさん。”水樹素子”と申します』
生気が入った顔になり、一言。彼女”水樹素子”は呟いた。
「お前には彼女の管理者になって欲しい」
それに補足するように祖父は蒼に
こうして蒼の面倒な日々は幕を開けた。


――しばしの沈黙。水樹はキョロキョロと周りを見ている。
沈黙を打ち破るように蒼が水樹に話しかけた。
「で、実際に君はなにができるの? 」
水樹は嬉しそうに、少しためてから口を開いた。
「私はなんでもできるわ! 正しい指示をもらえれば世界を大きく変化させることができるの、試しにやって見る? 」
したり顔で蒼をみる。そしてさっと振り向いて祖父にウインクを送った。
祖父はそれでわかったのか何か聞こえない声で何かを唱えた。
水樹は唱え終わる前に手を前に出し、指で輪っかを作った。蒼がそれを食い入るように見る。
すると指の真ん中に小さなひかる点が生まれ、どんどん大きくなってきた。そしていつのまにかこの部屋を真昼の外のように明るく照らすほど
大きい光になり3人を包み込む。
「うわっ、ストップ! ストップ! 」
蒼はそっさに目を多い、光から目を守りつつ、叫んだ。
「人間向けじゃないしな、水樹、止まれ」
祖父は目を覆いながらもこうなることを予測していたのか、軽い笑みを浮かべながら水樹を制止した。
「管理者、あなたのことはなんて呼べばいいの? 」 
光が消え、部屋を照らす灯りが先ほどの古ぼけた中央の明かりのみになり、明るい状態から暗くなったからだろう互いの顔が見えないほど暗くなった。
手を下ろし、蒼の方をじっと見つめながら水樹は質問する。
「緒岸 蒼言う名前だから……、 蒼でいいよ」
「蒼ね、よろしく」
「よろしく」
水樹は握手を求め、蒼は握手をした。
それを祖父は暖かい目で見守った。


次の日。緒岸は少し早めに家を出た。水樹素子も一緒だ。
「彼女と一緒に通学してくれ、準備はできるから」
「そんな急に言われても」
「とりあえず行った行った」
祖父から押し付けられるように一緒に行けと言われたからだ。
人が自分しかいない通りを二人で歩く。今日は昨日に比べて雲が多い。
「一緒に来ちゃったけど、どうするの? 籍もないし怪しまれない? 」
「問題ないわ、私はそもそも学校に行くわけではない、生徒ではないもの」
胸を張りなが、彼女は心配なさそうに言った。緒岸はそれを不安げに聞く。
「そういえば……」
水樹が立ち止まった。つられて緒岸も立ち止まる。
「私の姿、実は他の人間には見えてないの、正確には他の人間には干渉してないの。 だから、あなた今ずっと独り言喋ってるわよ 」
笑いをこらえながら、話す。緒岸はぽかんとなった。
「だから私はよく見ると浮いてるし、風の影響も受けるけど、風自体が弱まったりしない」
「……つまり、この世界に、君は干渉できない。一部の人だけが干渉できるって事? 」
「そんな感じ、理解が早くて助かるわ」
なんとか理解を整理して緒岸は話す。水樹は嬉しそうに返事をすると前を向いて歩き出した。
先ほどの話を聞いて、できる限り会話をしないでおこうと意識しながら緒岸は歩く。
いつもの賑やかな通学路に出るが確かに彼女を意識する人はいなかった。
いつもの時間より早く家を出たので、黒田とは合わず教室に着いた。
「じゃ、私は少し出かけてくるから、授業終わりに会いましょうか」 
いつもの時間より早く家を出たので、黒田とは合わず教室に着いた。
教室に入る直前、そう言い残して水樹は颯爽と生徒が行き交う廊下の中に消えて行った。
それと入れ違いに黒田が教室に入って来る。内心合わなくてよかったと安心する緒岸。
「おっす、なんかあった?始まる前から今日は終わったって感じだけど」
「色々未来の技術に触れたからね」
いつもより疲れているとは思ったが黒田にはバレたようだ。適当な返事を返しつつ授業に備える。
黒田は空気を察したのか、それ以上聞かずに席に戻った。


授業が終わり、黒田と別れ緒岸は水樹を探しに行った。
あてがあるわけではないので、適当に学内をうろつくだけだが。
各教室、特別教室、体育館、校庭……
様々な場所では様々な部活が活動していた。いつもは遠くの方から聞こえていた活気ある声が
どこから来ているのか、その疑問が一気に晴れて行く。緒岸としてはこの暑い中暑い場所で精を出す
のは辛そうだなと言う感情でしかなかったが、色々な人の部活を眺めるのは嫌いではなかった。
校庭の隅にあった茂った庭のような場所に水樹はまるで待っていたと言わんばかりの様子で立っていた。
「お疲れっ! ここまで結構かかったね」
「手がかりがないからね、でなんでこんなところにいるの? 」
水樹はとても元気そうに話しかけて来た。緒岸はなんでここにいるのかと疑問をぶつけた。
「それはね、彼に挨拶してたんだよ」
水樹は紹介するよと、後ろの茂った庭の奥にあった祠に緒岸を引っ張った。
急に腕を引っ張られ緒岸は困惑しつつも着いて行く。
「今日は不在だったから連絡用紙を渡すだけだけどね」
水樹は胸ポケットから小さなメモ切れを取り出し、緒岸に手渡した。
”
[base]
name=水樹素子
property=第五類外部資源活用型一般級
uuid=2770032032
admin=緒岸蒼
region=東日本
[Comment]
新規登録した神です。近年枯渇している第五類なので、是非とも活用いただけると。
資源元は水樹陽子なので、かなり大規模な事故にも対応できます。
よろしくお願いします。
”
「これは原本の写し、一応挨拶だから保管しとかないとね」
緒岸にはこのメモが何か設定を書いていることは推測できたが、よくわからない単語がたた出ており、完全には理解できなかった。
水樹は緒岸の手からメモを引っこ抜くとまた自分の胸ポケットにしまった。
「さて、部活へ行きましょうか」
水樹は緒岸の手を引っ張り、校舎への入り口を目指した。鞄を落としそうになるが水樹は気にしない。
とっさに緒岸は手を振り切り、質問を投げかける。
「みんなには見えないのに部活行ってもどうするんだ! 」
緒岸にはあの部室の中で、水樹がいたとしてしらを切り続けることが無理なことは分かっていた。
そもそもなんで部活を知っているのかに関しては今回は不問だとしても、この点だけははっきりさせておきたかった。
「大丈夫よ、公開範囲をいじって、部室の中の人には私を見せる予定だから」
水樹はさも当然のように答えた。緒岸には納得いかなかった。
「こう言うのってむやみに人に話していいものなの? 」
「蒼に近しい人なら大丈夫よ、5人にも満たないしね」
どうやや彼女には今日の部室の人数まで把握されているらしい。これはもう黙って着いて行ったほうがいいな、と半ば呆れながら
緒岸は彼女に着いて行った。遠くの方でサッカー部の元気な声が小さく聞こえた。

部活中の廊下をスイスイ進む彼女はさながらベテラン3年生のような足取りだった。
ひょっとしたら今までで最短なんじゃないか、と緒岸が考えていると目の前が部室だった。
「さ、中に入って」
水樹はドアの前で、そっと緒岸を待った。ドアを開けるのはまだできないらしい。緒岸はドアを開け中に入った。
「お疲れ」
「遅いな〜」
「……」
中に入ると、美澄が返事をくれた。黒田は夢中なのか返事がなかった。
緒岸は部室の中はいつもと一緒だ、安心しながら中に入る。その後ろを水樹が堂々と中に入る。歩き方に遠慮がない。
緒岸がいつもの席につくと、鞄からパソコンを取り出した。いつものように作業を始めるか、とした時水樹から耳打ちをされた。
(紹介したい物があると言って鞄の中に入れて置いた、メモ用紙をとってくれない?)
(鞄の中って言ってもわからないと思うよ)
(数学のノートのはじめの方、見つけたら読み上げて)
(了解)
緒岸はそっと鞄の中を漁りだした。他の二人は特に興味ないのか自分の作業に打ち込んでいる。
数学のノートを開くと確かに1ページの所に和紙でできたメモ紙が挟まっていた。数学は最後の授業だったし一体どこで入れたのだろう、
そう不審がりながらもメモを取り出した。何かボールペンで書かれている。なぜか紙は和紙なのに文字はボールペンだった。
「ちょっと話したいことがあるんだけど」
二人の作業を止め、こっちに注目を集めた。二人はこっちをみる。緒岸は質問が来る前にメモを読み上げた。
「****」
読み上げた直後、水樹素子が二人の前に現れた。緒岸には実感が湧かなかったが目の前の二人の反応を見るに
無事見えるようになったのだろう。二人は目を丸くしている。
「えっ? えっ? えっ?? 」
「なにこれなにこれ? なにこれ? 」
当然の反応だろう、緒岸はそう思いつつも説明が大変そうだなと、ため息。
水樹は動くとややこしくなることがわかっていたのだろう、胸を張っているポーズで立ち止まっていた。正直とてもシュールな光景だったが
突っ込むことをため、3人を遠目で見ることにした。
相変わらず二人は水樹に興味津々だ。立ち上がって水樹を中心に集まった。
「これなに? 新しいホログラム? 」
「新しいガジェットにしては凄すぎない? 」
緒岸に質問をして来るが緒岸は答えず、水樹にアイコンタクトをしただけだった。
水樹はそれを受け取り、ずっと胸を張っているだけだったが動きだした。
「そんなに私が珍しい? 私の名前は水樹素子、いわゆる神さまってやつよ! 新しく緒岸の下で動くことになったからよろしく!」
「うわ喋った」
「動くよ! 動くよ! 」
「……失礼ね。 動くわ! 」
二人の対応がどっかの会社が発表したロボットのような扱いで、ムッときたのか水樹は謎の踊りを始めた。
謎の踊りをしたところで二人の印象は変わらず”謎の動くロボット”だった。
「よくわかってないんだけど、彼女は神さまらしい。さっき見たでしょ? いきなり浮かび上がったやつ。 あれは神の力ってやつらしい」
緒岸が諦めたように、場を沈めるように言った。二人は緒岸が言うならばそうなのか、と少しだけ信じた。

夜、3人+1体で学校に残り、例の場所に行き特定のイベントを発生させる。
持ち物はモバイルバッテリー、手回し発電機、LANケーブル、食料……
誰にも認知されていないタイミングで古い学校のような場所に転送される。
古い学校に飛ばされ、インターネットと隔離される。
目視ての判断だが、現世の学校とは違うと推定される。
電気がついているので電源供給はされてそう。コンセントにACアダプタを挿してみる、供給はされている。テスターを持ってくるべきだったね
一人がそうしている間、他の二人の興味は窓へ向かう。窓の外は真っ暗。持っていたモバイルバッテリー(一番硬い物質と判断されたため)で叩いてみる。
割れない。傷一つつかない。水樹の分析により、外力から保護されているという事実が出てくる。
水樹の力を使い、破壊してみる提案をしてみるが、個々は水樹の管理領域ではないので分析しか出来ないと言われる。
水樹から事実を述べられる。事実として
- 時間軸は同じ。進む速度も同じ。飛ばされてからN分たったらN分現実世界でも進む世界。
- 圏外であり、仮想空間である可能性が高い
- 作者は神ではない、又は未登録神だが、能力的に神程度の支援がないと不可能
この事実を聞いて、Oが時間軸が同じ事から行方不明者が生存している可能性を提示、捜索を提案。全員了承
その他のタスクとして作者に会い、動機を聞き出しつつ現世への転送を行う。
捜索を試みる。定期的に不気味な音が聞こえる。ゲームの世界だけだと思ったのに……不気味
壁やドアを触りながら、水樹がこの世界の物質は変形をしない物になっていると仮設を立てる。スライド式のドアは閉まるが、チョークは使えない(削れないため)
教室の名前は「1−1」「1−2」「1−3」……
階段を見つける。
上に行くか、下に行くか。個々は仮想空間なのでどちらも変わりないのでは?今の階が1なので上のほうが部屋の数多そう。皆了承。上に向かう
階段の途中、上の階から覗く影を見つける。こっちが正しいっぽい。
上の階につく、教室が見れる、「2−3」階段「2−4」
確かに2年生のようだ、全容を知るためにこのまま上に行ってみよう。3階(仮)に向かう。上に階段が続く。
「3−3」階段「3−4」上に階段が続く。
「4−3」階段「4−4」上に階段が続く。おっとこれはおかしいな、学年が修業年限に準拠してないぞ。机は大きいけど小学校なのかな?
「5−3」階段「5−4」上に階段が続く。
「6−3」階段「6−4」上に階段が続く。
「7−3」階段「7−4」上に階段が続く。流石におかしいね、多分だけどこれ終わらない気がする。水樹が自分分析ならできると、指示を乞う。
もっと早く言ってくれれば、こんなに階段登らなくても良かったのに。いや分析は地形だけで人の存在はわからない。なるほど、わかり。Oが指示し、水樹が分析する。
こりゃすごいわ、上にも下にも終わりはないっぽい。1000ー1もあるし下に関しては−103−1になる。1が初めではないっぽい。
しかも横に関しても終わりがなく「1−1000」もある。流石にマイナスはないけど終わりがない。
あと、教室の合間に特別教室、トイレ、階段、水道等がランダムに配置されている。階段も一本で全部の階に行けるわけではなく、25階までしか無いっぽい。
現在の分析の中で特異的な部屋はある?お待ちを。
行方不明者は最初の場所からそう離れないはず、なので戻ろうか。うむ
「1−1」に戻ってみた。作戦会議だ。



\end{document}
